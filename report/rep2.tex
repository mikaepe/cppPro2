% REPORT FOR PROJECT2 SF2565

\documentclass[a4paper,10pt]{article}

%%Packages	%%%	%%%	%%%	%%%	%%%	%%%	%%%
\usepackage{amsmath,framed}
\usepackage[latin1]{inputenc} 	
\usepackage{listings}
\usepackage{xcolor}
\usepackage{graphicx}
\usepackage{placeins}

%%Settings	%%%	%%%	%%%	%%%	%%%	%%%	%%%
\renewcommand{\d}{\text{d}}
\newcommand{\e}{\text{e}}
\newcommand{\ve}{\mathbf}
\newcommand\numb{\addtocounter{equation}{1}\tag{\theequation}}

\setlength\fboxsep{1.2mm}
\setlength\fboxrule{0.5mm}

\lstset { %
  language=C++,
  backgroundcolor=\color{black!3},	% set backgroundcolor
  basicstyle=\footnotesize,		% basic font setting
}

%%Margins	%%%	%%%	%%%	%%%	%%%	%%%	%%%
\usepackage{geometry}
\geometry{
  a4paper,
  left=35mm,
  right=35mm,
  top=35mm,
  bottom=35mm,
}

%%Header & Footer	%%%	%%%	%%%	%%%	%%%	%%%
\usepackage{fancyhdr}
\pagestyle{fancy}
\renewcommand{\headrulewidth}{1pt}
\rhead{Hanna Hultin, Mikael Perssson}
%8509080456 mitt persnr om vi vill ha det
\lhead{P2 SF565}

\title{Project 2, SF2565}
\author{Hanna Hultin, hhultin@kth.se, 931122-2468, TTMAM2 \\ Mikael Persson, mikaepe@kth.se, 850908-0456, TTMAM2}
\begin{document}
\maketitle

\subsubsection*{Task 1}
In this task we implemented the Taylor series for $\exp (x)$, that is 
\begin{equation*}
  \exp(x) = \sum_{n=0}^\infty \frac{x^{n}}{n!}.
\end{equation*}
This was done by a straight-forward implementation of the 
sum which included all terms until the first term less than \texttt{tol}.
To validate our routine we check it against the exponential function from the 
standard library (i.e. \texttt{cmath}) for some values of $x$. 
The results are shown in Table \ref{TABtask1}.
An alternative to implement the series by summing term by term is to use Horners algorithm, 
a problem with this approach however is to get an estimate of the error. This is due
to the fact that this algorithm 
needs to have $N$ from the beginning to get an estimate of the error one would have to compute the
whole series for some $N$ and then recompute it for $N+1$.

\begin{table}[!ht]
\centering 
  \begin{minipage}[t]{105mm}
    \caption{
      Performance of the Taylor series implementation shown for selected values of 
      $x$. The tolerance used was set to \texttt{tol = 1e-10}. 
    } 
    \label{TABtask1}
  \end{minipage}

  \vspace{5mm}
  \begin{tabular}{l l l l l l} 
    \texttt{ }&\texttt{x = -1}& \texttt{x = 0} & \texttt{x = 1} & \texttt{x = 5} & \\
    \hline
    \texttt{cmath}	& 0.367879441171 & 1.000\dots & 2.718281828459 & 148.413159102577 \\
    \texttt{myexp}	& 0.367879441172 & 1.000\dots & 2.718281828458 & 148.413159102572 \\
    \vspace{-2mm} \\ \hline
    \texttt{ }& \multicolumn{2}{l}{\texttt{x = 10}} & \multicolumn{2}{l}{\texttt{x = 15}} \\
    \texttt{cmath}	& \multicolumn{2}{l}{22026.465794806718} & 
    \multicolumn{2}{l}{3269017.372472110670}\\
    \texttt{myexp}	& \multicolumn{2}{l}{22026.465794806707} & 
    \multicolumn{2}{l}{3269017.372472110670}\\
     \vspace{-2mm} \\ \hline
    \texttt{ }& \multicolumn{2}{l}{\texttt{x = 20}} & \\
    \texttt{cmath}	& \multicolumn{2}{l}{485165195.409790277481} \\
    \texttt{myexp}	& \multicolumn{2}{l}{485165195.409790456295} \\
  \end{tabular}
\end{table}


\noindent
The program for task 1 is used, and gives outputs, as follow (note that the user has to input $x$):
\begin{center}
\begin{minipage}[t]{85mm}
\begin{lstlisting}
user@computer:~$ ./t1 
x = 1.75
iterations performed (myexp): 17
cmath:  exp(x) = 5.754602676006
myexp:  exp(x) = 5.754602676002
user@computer:~$ 
\end{lstlisting}
\end{minipage}
\end{center}

\subsubsection*{Task 2}
In the second task we implemented the Taylor series for 
the exponential of a square matrix $A$, that is 
\begin{equation*}
  \exp(A) = \sum_{n=0}^\infty \frac{A^{n}}{n!}.
\end{equation*}
To be able to compute the sum we implemented a class for square matrices. The class has as private members the dimension $m$ of the square matrix and $a$ which is a pointer to an array of pointers to arrays each containing a row of the matrix. 
We have the following constructors:
\begin{itemize}
  \item \texttt{Matrix()} the default constructor
  \item \texttt{Matrix(int mm)} initializes a square matrix of dimension mm filled with zeros
  \item \texttt{Matrix(const Matrix \&M)} copy constructor which does a deep copy of the Matrix M
\end{itemize}
We also have a destructor which deletes $a$.
To be able to compute our sum we did the following operator overloadings: 
\begin{itemize}
  \item \texttt{Matrix \&operator=(const Matrix \&M)} equates Matrix to M by deep copy
  \item \texttt{Matrix \&operator+=(const Matrix \&M)} adds elements of M to matrix
  \item \texttt{Matrix \&operator-=(const Matrix \&M)} subtracts elements of M to matrix
  \item \texttt{Matrix \&operator*=(const Matrix \&M)} performs matrix multiplication with matrix M and substitutes the original matrix with the new product
  \item \texttt{Matrix \&operator*=(const double d)} performs 
    multiplication with scalar d for each element in matrix
  \item \texttt{Matrix \&operator*(const Matrix \&M)} performs matrix multiplication with matrix M and returns this as a new matrix
\end{itemize}
Finally we have some functions for our matrix class:
\begin{itemize}
  \item \texttt{void fillMatrix(double b[])} fills the matrix with the elements in b[] column wise
  \item \texttt{void identity()} transforms the matrix to an identity matrix
  \item \texttt{void printMatrix()} prints the matrix using std::cout
  \item \texttt{double norm()} computes and returns the infinity norm of the matrix
  \item \texttt{int sizeMatrix()} returns the size of the matrix
\end{itemize}

When we had implemented our class we computed the Taylor series similar to the 
implementation in task 1. We included all terms of the sum until the 
first term with a norm smaller than \texttt{tol}.
To validate our routine we check it against the p
rovided routine, \texttt{r8mat\_expm1.cpp}.
Some results are shown in \ref{TABtask2}.


\begin{table}[!ht]
\centering 
  \begin{minipage}[t]{105mm}
    \caption{
      Performance of the matrix Taylor series implementation shown for some matrices $A$. 
      The tolerance used was set to \texttt{tol = 1e-10}. 
    } 
    \label{TABtask2}
  \end{minipage}

  \vspace{6mm}
  \begin{tabular}{l l l l l l} 
    \texttt{ }&$A = 
    \left[ \begin{array}{cc}
      1 & 0 \\
      0 & -1 
    \end{array}
    \right]$
    & $A = 
    \left[ \begin{array}{cc}
      0 & 10 \\
      0 & 0 
    \end{array}
    \right]$ &  $A = 
    \left[ \begin{array}{cc}
      1 & 2 \\
      5 & 6 
    \end{array}
    \right]$  \vspace{2mm} \\
    \hline \\ 
    \texttt{r8mat}	& 
    $\left[
      \begin{array}{cc}
	2.17828 & 0.00000 \\
	0.00000 & 0.36788
      \end{array}
    \right]$ & 
    $\left[
      \begin{array}{cc}
	1.00000 &10.00000 \\
	0.00000 & 1.00000
      \end{array}
    \right]$ &
    $\left[
      \begin{array}{cc}
	354.70420 & 462.55578 \\
	1156.38945 & 1511.09365
      \end{array}
    \right]$\vspace{2mm} \\
    \texttt{myexp}	& 
    $\left[
      \begin{array}{cc}
	2.17828 & 0.00000 \\
	0.00000 & 0.36788
      \end{array}
    \right]$ &
    $\left[
      \begin{array}{cc}
	1.00000 &10.00000 \\
	0.00000 & 1.00000
      \end{array}
    \right]$ & 
    $\left[
      \begin{array}{cc}
	354.70420 & 462.55578 \\
	1156.38945 & 1511.09365
      \end{array}
    \right]$ 
    \vspace{2mm} \\
    $||\texttt{r8mat-myexp}||_\infty$ & 0.000000000000816 & 
     0.000000000000000 &  0.000000000018190 \\
    \vspace{-2mm} \\ \hline \\
    \texttt{ }&\multicolumn{3}{l}{$A = 
    \left[ \begin{array}{cccc}
      1 & 0 & 0 & 0\\
      0 & -1 & 0 & 0 \\
      0 & 0 & 5 & 0 \\
      0 & 0 & 0 & 10 
    \end{array}
    \right]$}  \vspace{2mm} \\
    \hline \\ 
    \texttt{r8mat}	& 
    \multicolumn{3}{l}{
    $\left[
      \begin{array}{cccc}
	2.17828 & 0.00000 & 0.00000 & 0.00000 \\
	0.00000 & 0.36788 & 0.00000 & 0.00000 \\
	0.00000 & 0.00000 & 148.41316 & 0.00000 \\
	0.00000 & 0.00000 & 0.00000 & 22026.46579
      \end{array}
    \right]$ } \vspace{2mm} \\
    \texttt{myexp}	& 
    \multicolumn{3}{l}{
    $\left[
      \begin{array}{cccc}
	2.17828 & 0.00000 & 0.00000 & 0.00000 \\
	0.00000 & 0.36788 & 0.00000 & 0.00000 \\
	0.00000 & 0.00000 & 148.41316 & 0.00000 \\
	0.00000 & 0.00000 & 0.00000 & 22026.46579
      \end{array}
    \right]$} \vspace{2mm} \\
    $||\texttt{r8mat-myexp}||_\infty$ & 0.000000000003638  \\
    \vspace{-2mm} \\ \hline
  \\
  \texttt{ }&\multicolumn{3}{l}{$A = 
    \left[ \begin{array}{cccc}
      1 & 5 & 10 \\
      -1 & -1 & -1 \\
      1 & 1 & 1 
    \end{array}
    \right]$}  \vspace{2mm} \\
    \hline \\ 
    \texttt{r8mat}	& 
    \multicolumn{3}{l}{
    $\left[
      \begin{array}{cccc}
	9.99487 &  23.07888 & 40.68389 \\
	 -3.52100 & -6.90010 & -13.37397 \\
	  3.52100 &  7.90010 & 14.37397
      \end{array}
    \right]$ } \vspace{2mm} \\
    \texttt{myexp}	& 
    \multicolumn{3}{l}{
    $\left[
      \begin{array}{cccc}
	9.99487 &  23.07888 & 40.68389 \\
	 -3.52100 & -6.90010 & -13.37397 \\
	  3.52100 &  7.90010 & 14.37397
      \end{array}
    \right]$} \vspace{2mm} \\
    $||\texttt{r8mat-myexp}||_\infty$ & 0.000000000004794  \\
    \vspace{-2mm} \\ \hline

\end{tabular}
\end{table}

\FloatBarrier



\noindent
An example of the program output (again, the user has to input the size of the matrix and then the rows of the matrix) is given below:
\begin{center}
\begin{minipage}[t]{85mm}
\begin{lstlisting}
user@computer:~$ ./t2 
Size of square matrix: 2
(elements separated by blanks)
row 1:
1 2
row 2:
3 4
A = 
[1, 2
 3, 4]
given function: exp(A) = 
[51.96896, 74.73656
 112.10485, 164.07380]
own implementation: exp(A) = 
[51.96896, 74.73656
 112.10485, 164.07380]
norm expA-myexpA = 0.000000000008299
user@computer:~$ 
\end{lstlisting}
\end{minipage}
\end{center}


\newpage
\subsubsection*{Code for task 1}
\lstinputlisting[language=C++,basicstyle=\scriptsize]{../t1.cpp}


\newpage
\subsubsection*{Code for task 2}
\lstinputlisting[language=C++,basicstyle=\scriptsize]{../matrix.hpp}
\lstinputlisting[language=C++,basicstyle=\scriptsize]{../t2.cpp}

\end{document}





