% REPORT FOR PROJECT2 SF2565

\documentclass[a4paper,10pt]{article}

%%Packages	%%%	%%%	%%%	%%%	%%%	%%%	%%%
\usepackage{amsmath,framed}
\usepackage[latin1]{inputenc} 	
\usepackage{listings}
\usepackage{xcolor}
\usepackage{graphicx}
\usepackage{placeins}

%%Settings	%%%	%%%	%%%	%%%	%%%	%%%	%%%
\renewcommand{\d}{\text{d}}
\newcommand{\e}{\text{e}}
\newcommand{\ve}{\mathbf}
\newcommand\numb{\addtocounter{equation}{1}\tag{\theequation}}

\setlength\fboxsep{1.2mm}
\setlength\fboxrule{0.5mm}

\lstset { %
  language=C++,
  backgroundcolor=\color{black!3},	% set backgroundcolor
  basicstyle=\footnotesize,		% basic font setting
}

%%Margins	%%%	%%%	%%%	%%%	%%%	%%%	%%%
\usepackage{geometry}
\geometry{
  a4paper,
  left=35mm,
  right=35mm,
  top=35mm,
  bottom=35mm,
}

%%Header & Footer	%%%	%%%	%%%	%%%	%%%	%%%
\usepackage{fancyhdr}
\pagestyle{fancy}
\renewcommand{\headrulewidth}{1pt}
\rhead{Hanna Hultin, Mikael Perssson}
%8509080456 mitt persnr om vi vill ha det
\lhead{P2 SF565}

\title{Project 2, SF2565}
\author{Hanna Hultin, hhultin@kth.se, 931122-2468, TTMAM2 \\ Mikael Persson, mikaepe@kth.se, 850908-0456, TTMAM2}
\begin{document}
\maketitle

\subsubsection*{Task 1}
In this task we implemented the Taylor series for $\exp (x)$, that is 
\begin{equation*}
  \exp(x) = \sum_{n=0}^\infty \frac{x^{n}}{n!}.
\end{equation*}
This was done by a straight-forward implementation of the 
sum which included all terms until the first term less than \texttt{tol}.
To validate our routine we check it against the exponential function from the 
standard library (i.e. \texttt{cmath}) for some values of $x$. 
The results are shown in Table \ref{TABtask1}.
An alternative to implement the series by summing term by term is to use Horners algorithm, 
a problem with this approach however is to get an estimate of the error. Since this algorithm 
needs to have $N$ from the beginning to get an estimate of the error one would have to compute the
whole series for some $N$ and then recompute it for $N+1$. For reference we included 
a series implemented with Horners algorithm using constant $N = 40$. This is 
\texttt{myexpH} in Table \ref{TABtask1}.

\begin{table}[!ht]
\centering 
  \begin{minipage}[t]{105mm}
    \caption{
      Performance of the Taylor series implementation shown for selected values of 
      $x$. The tolerance used was set to \texttt{tol = 1e-10}. 
    } 
    \label{TABtask1}
  \end{minipage}

  \vspace{5mm}
  \begin{tabular}{l l l l l l} 
    \texttt{ }&\texttt{x = -1}& \texttt{x = 0} & \texttt{x = 1} & \texttt{x = 5} & \\
    \hline
    \texttt{cmath}	& 0.367879441171 & 1.000\dots & 2.718281828459 & 148.413159102577 \\
    \texttt{myexp}	& 0.367879441172 & 1.000\dots & 2.718281828458 & 148.413159102572 \\
    \texttt{myexpH}	& 0.367879441171 & 1.000\dots & 2.718281828459 & 148.413159102577 \\
    \vspace{-2mm} \\ \hline
    \texttt{ }& \multicolumn{2}{l}{\texttt{x = 10}} & \multicolumn{2}{l}{\texttt{x = 15}} \\
    \texttt{cmath}	& \multicolumn{2}{l}{22026.465794806718} & 
    \multicolumn{2}{l}{3269017.372472110670}\\
    \texttt{myexp}	& \multicolumn{2}{l}{22026.465794806707} & 
    \multicolumn{2}{l}{3269017.372472110670}\\
    \texttt{myexpH}	& \multicolumn{2}{l}{22026.465794806280} & 
    \multicolumn{2}{l}{3269017.295864568558}\\
     \vspace{-2mm} \\ \hline
    \texttt{ }& \multicolumn{2}{l}{\texttt{x = 20}} & \\
    \texttt{cmath}	& \multicolumn{2}{l}{485165195.409790277481} \\
    \texttt{myexp}	& \multicolumn{2}{l}{485165195.409790456295} \\
    \texttt{myexpH}	& \multicolumn{2}{l}{485152859.445135474205} \\ 
  \end{tabular}
\end{table}



\begin{table}[!ht]
\centering 
  \begin{minipage}[t]{105mm}
    \caption{
      Result for some selected values of $x$ and $N$.
    } 
    \label{TABtask12}
  \end{minipage}

  \vspace{5mm}
  \begin{tabular}{l l l l} 
    \texttt{x}&\texttt{N}&\texttt{sin err/N+1term} & \texttt{cos err/N+1term} \\
    \hline
    \texttt{-1}	& \texttt{1}	& \texttt{0.97651818} & \texttt{0.96725534} 	\\
    \texttt{-1}	& \texttt{2}	& \texttt{0.98623657} & \texttt{0.98233977} 	\\
    \texttt{-1}	& \texttt{5}	& \texttt{0.99525535} & \texttt{0.99452832} 	\\
    \texttt{-1}	& \texttt{10}	& \texttt{nan} & \texttt{nan} 	\\
    \hline 
    \texttt{1}	& \texttt{2}	& \texttt{0.98623657} & \texttt{0.98233977} 	\\
    \texttt{1}	& \texttt{5}	& \texttt{0.99525535} & \texttt{0.99452832} 	\\
    \hline
    \texttt{2}	& \texttt{2} 	& \texttt{0.94641382} & \texttt{0.93165191}	\\
    \texttt{2}	& \texttt{5}	& \texttt{0.98122925} & \texttt{0.97838354}	\\
    \hline
    \texttt{3}	& \texttt{3}	& \texttt{0.92270632} & \texttt{0.90649330}	\\
    \texttt{3}	& \texttt{5}	& \texttt{0.95852439} & \texttt{0.95235057}	\\
    \texttt{3}	& \texttt{10}	& \texttt{0.98513048} & \texttt{0.98392664}	\\
    \hline
    \texttt{5}	& \texttt{3}	& \texttt{0.80518429} & \texttt{0.76830018}	\\
    \texttt{5}	& \texttt{5}	& \texttt{0.89113977} & \texttt{0.87584989}	\\
    \texttt{5}	& \texttt{10}	& \texttt{0.95977271} & \texttt{0.95639607}	\\
    \hline
    \texttt{10}	& \texttt{3}	& \texttt{0.47425379} & \texttt{0.41141849}	\\
    \texttt{10}	& \texttt{5}	& \texttt{0.65781950} & \texttt{0.62076516}	\\
    \texttt{10}	& \texttt{10}	& \texttt{0.85443812} & \texttt{0.84340922}	\\
    \texttt{10}	& \texttt{15}	& \texttt{0.92187411} & \texttt{0.91747435}	\\
    \texttt{10}	& \texttt{20}	& \texttt{0.94971903} & \texttt{0.94954946}	\\
  \end{tabular}
\end{table}

\FloatBarrier
\newpage
The output of the program used in task 1 is of the following form, note that the user has to input
the values of interest ($x$ and $N$).
\begin{center}
\begin{minipage}[t]{85mm}
\begin{lstlisting}
user@computer:~$ ./pro1_t1 
x = -1
N = 1
cmath:   sin(x) = -0.84147098
Taylor:  sin(x) = -0.83333333
cmath:   cos(x) = 0.54030231
Taylor:  cos(x) = 0.50000000
sin N+1-term = -0.00833333
cos N+1-term = 0.04166667
sin err/N+1term = 0.97651818
cos err/N+1term = 0.96725534
user@computer:~$ 
\end{lstlisting}
\end{minipage}
\end{center}

\subsubsection*{Task 2}
In the second task we implemented the Taylor series for the exponential of a square matrix $A$, that is $\exp(A) = \sum_{n=0}^\infty \frac{A^{n}}{n!}$. 

To be able to compute the sum we implemented a class for square matrices. The class has as private members the dimension $m$ of the square matrix and $a$ which is a pointer to an array of pointers to arrays each containing a row of the matrix. 

We have the following constructors:
\begin{itemize}
\item Matrix(): the default constructor
\item Matrix(int mm): initializes a square matrix of dimension mm filled with zeros
\item Matrix(const Matrix \&M): the copy constructor which does a deep copy of the Matrix M
\end{itemize}
We also have a destructor which deletes $a$.

To be able to compute our sum we did the following operator overloadings: 
\begin{itemize}
\item Matrix \&operator=(const Matrix \&M): equates Matrix to M by deep copy
\item Matrix \&operator+=(const Matrix \&M): adds elements of M to matrix
\item Matrix \&operator*=(const Matrix \&M): performs matrix multiplication with matrix M and substitutes the original matrix with the new product
\item Matrix \&operator*=(const double d): performs multiplication with scalar d for each element in matrix
\item Matrix \&operator*(const Matrix \&M): performs matrix multiplication with matrix M and returns this as a new matrix
\end{itemize}

Finally we have some functions for our matrix class:
\begin{itemize}
\item void fillMatrix(double b[]): fills the matrix with the elements in b[] column wise
\item void makeMatrix(): prompts the user to fill the rows of the matrix
\item void identity(): transforms the matrix to an identity matrix
\item void printMatrix(): prints the matrix using std::cout
\item double norm(): computes and returns the infinity norm of the matrix
\end{itemize}

When we had implemented our class we computed the Taylor series similar to the implementation in task 1. We included all terms of the sum until the first term with a norm smaller than $tol$.

To validate our routine we check it against the provided routine, \verb|r8mat_expm1.cpp|. The results are shown in \ref{TABtask2}.

\newpage
\subsubsection*{Code for task 1}
\lstinputlisting[language=C++,basicstyle=\scriptsize]{../t1.cpp}


\newpage
\subsubsection*{Code for task 2}
%\lstinputlisting[language=C++]{../matrix.hpp}
%\lstinputlisting[language=C++]{../t2.cpp}

\end{document}





